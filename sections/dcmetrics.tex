\subsection{Class Diagram Metrics}

% -- estou a incluir aqui as métricas q temos no nosso artigo, + notas sobre o trabalho d Michele Marchesi

These diagrams are used to describe the types of objects in a system and their relationships.
They describe the structure of a system by showing the system classes, their attributes, methods and the relationships between them.
Their quality have a huge impact on the final quality of the software under development, as they describe the general model of the system information.

Measures like the \emph{Number of Attributes in the Class} (NAC), the \emph{Number of Operations in the Class} (NOC), \emph{Number of Inherited Attributes } (NIA), \emph{Number of descents/ancestors od a Class}(NDC/NAC), or even the \emph{Number of Interfaces Implemented} (NII) are used both for indicate the system complexity and as a index of quality.\\

%In an ideal system, the interdependencies among packages and classes must be minimal

--- Citar mais trabalhos sobre DC com+exemplos de métricas %avaliar diversas refs q encontrei

 --- Falta Introduzir textualmente o trabalho d Marchesi nesta área para contextualizar as tabelas;


\begin{table}

\begin{minipage}[b]{0.5\linewidth}\centering
\begin{tabular}{ p{1,5cm} | p{4cm}}
\multicolumn{2}{l}{\textbf{Marchesi Metrics}} \\ \hline
\textbf{Metric} & \textbf{Description} \\ \hline
NC & Number of Classes \\ \hline
CL1 & Weighted Number of responsibilities of a Class   \\ \hline 
CL2 & Weighted Number of Dependencies of a Class \\ \hline 
CL3 & Depth of inheritance tree \\ \hline 
CL4 & Number of immediate subclasses of a given class \\ \hline 
CL5 & Number of distinct class \\ \hline 
\end{tabular}
\caption{\small{Marchesi Class Diagram Metrics}}
\label{t:dcm}
%\end{table}
%\end{center}
\end{minipage}
\hspace{0.3cm}
\begin{minipage}[b]{0.5\linewidth}
\centering

%\begin{center}
%\begin{table}[h]
\begin{tabular}{ p{1,5cm} | p{4cm}}
\multicolumn{2}{l}{\textbf{Marchesi Metrics}} \\ \hline
\textbf{Metric} & \textbf{Description} \\ \hline
NP & Number of Packages. \\ \hline 
PK1 & Number of Classes \\ \hline
PK2 & Weighted Number of responsibilities of a Class   \\ \hline 
PK3 & Overall Coupling among Packages \\ \hline 
\end{tabular}
\caption{\small{Marchesi Package Metrics}}
\label{t:pcm}
%\end{table}
%\end{center}
\vspace{0.78cm}
\end{minipage}
\end{table}




