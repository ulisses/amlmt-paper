\section{Applying Metrics To UML Models}\label{metrics}
An \umlS model can be made from different diagrams, each one with a distinct view of the system. Use Cases diagrams expose
the functional requirements the system have and how each user intercats with them. They are a good overview in what
features the system offers to the end user.\\
Class Diagrams are more the blueprint of the application under the developer prespective. It tell the programming components a system has and
how they related to each other.\\
Package Diagrams describe how we group the classes and how these groups relate to each other (\textit{package import}, \textit{package merge}).\\
\indent We believe that an \umlS diagram need to have at least these three diagrams implemented, beacuse they give to both the costumer and the developer
the full knowledge of the system.\\
\indent Here we present some metrics related to this three fundamental diagrams and we terminate this section by explaining some metrics to other \umlS diagrams.

