\section{Applying Metrics To \uml\ Models}\label{metrics}

An \uml\ model can be made from different diagrams, each one with a distinct view of the system.
We have, in one hand, Use Cases diagrams which expose the system functional requirements and how each user interacts with them. They are a good overview of what features the system offers to the end user.
In the other hand, we use Class Diagrams for represent the blueprint of the application under the developer perspective: they illustrate which programming components a system has and how they related to each other.
Package Diagrams describe how to group the classes and how these groups relate to each other (\emph{package import}, \emph{package merge}).
%We believe that an \uml\ diagram need to have at least these three diagrams implemented, because they give to both the costumer and the developer the full knowledge of the system.
Here we present some metrics related to this three fundamental diagrams and conclude the section by introducing some metrics for other not less important \uml\ diagrams.

