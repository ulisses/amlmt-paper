\section{Metrics Assessment} \label{assess}
\begin{comment}
Effective management of any process requires quantification, measurement, and modeling.
Software metrics provide a quantitative basis for the development and validation of models of the software development process.
Metrics can be used to improve software productivity and quality\cite{g1:Millis:1998}.
\end{comment}



%And so, we need to model complex sistems. 
When interpreting the values obtained from measurements, one might question: \emph{is the metric really measuring the intended attribute?} This is a question that is present in the industry, yet unsuccessfully answered. 

Working with models, one might want to know the quality of its model, i.e., which amount of the model really reflects object proprieties. To discuss model quality, one must use metrics to quantify those proprieties. Fenton~\cite{g1:Fenton:1999} estimates that companies spend about 4\% of the development budget in the establishment of metrics programs, therefore, engineers should also certify and guarantee that the applied metrics actually quantify, measure, and model the attributes of the system.

Kaner and Bond~\cite{g1:kaner:2004} proposed a framework for metric evaluation, saying that if a project or company is managed according to the results of inadequately validated metrics,and  not tightly linked to the attributes they are intended to measure, distortions and disfunctionalities should be commonplace.

\begin{comment}
	This has a likely consequence: if a project or company is managed according to the results of measurements, and those metrics are inadequately validated, insufficiently understood, and not tightly linked to the attributes they are intended to measure, measurement distortions and dysfunctional should be commonplace\cite{g1:kaner:2004}.
\end{comment}

The industry, although not having a formal answer to this question, has advanced some steps forward in this direction. The IEEE Standard 1061~\cite{g1:Ieee1061:1998} defines an \emph{attribute} as \emph{``a measurable physical or abstract property of an entity''}. A \emph{quality factor} is a type of attribute,
\emph{``a management-oriented attribute of software that contributes to its quality''}. A \emph{metric} is defined as being a {\bf measurement function}, and a {\bf software quality metric} is defined as \emph{``a function whose inputs are software data and whose output is a single numerical value that can be interpreted as the degree to which software possesses a given attribute that affects its quality''}. Any software metric must comply with the following criteria: correlation, consistency, tracking, predictability, discriminative power and reliability.
This provides a sound layout of a methodology for developing metrics for software quality attributes.

%* explicar alguns critérios\\
%* ex(a correlação dá isto, enquanto que o tracking dá-nos isto)\\
