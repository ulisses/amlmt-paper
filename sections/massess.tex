\section{Metrics Assessment} \label{assess}




%--Entra o assunto da Secção 2 do artigo do Azevedo, JJ e Tiago.

    * Explicar aqui que as métricas não podem ser observadas, por si só, fora do contexto. Este problema prende-se essencialmente com o facto deste tipo de medições ser empírica, ou baseada em métodos empíricos.

\begin{quotation}
	Effective management of any process requires quantification, measurement, and modeling. Software metrics provide a quantitative basis for the development and validation of models of the software development process. Metrics can be used to improve software productivity and quality.~\cite{g1:Millis:1998}
\end{quotation}

    * Explicar em maior detalhe o que se entendo como validação de métricas (Kaner e Walter Bond).


\par Fenton~\cite{g1:Fenton:1999} estimates that companies spend about 4\% of the development budget in the establishment of metrics program, therefore, engineers should also certify and garantee that the applyed metrics actualy quantify, measure, and model the atributes of the system.
\begin{quotation}
	This has a likely consequence: if a project or company is managed according to the results of measurements, and those metrics are inadequately validated, insuficiently understood, and not tightly linked to the attributes they are intended to measure, measurement distortions and dysfunctional should be commonplace.~\cite{g1:kaner:2004}
\end{quotation}


\par The IEEE Standard 1061~\cite{g1:Ieee1061:1998} lays out a methodology for developing metrics for software quality attributes. The standard defines an attribute as \emph{``a measurable physical or abstract property of an entity.''} A quality factor is a type of attribute, \emph{``a management-oriented attribute of software that contributes to its quality.''}

\par A metric is defined as being a {\bf measurement function}, and a {\bf software quality metric} is defined as \emph{``a function whose inputs are software data and whose output is a single numerical value that can be interpreted as the degree to which software possesses a given attribute that affects its quality.''}

\par Any software metric must comply with the following criteria: correlation, consistency, tracking, predictability, discriminative power and reliability.