\section{Tools for UML Metrics Calculation} \label{tools}

Nowadays, the most popular UML tools for software application development are \textsf{Visual Paradigm for UML\footnote{Available at \url{http://www.visual-paradigm.com/product/vpuml/}}} and \textsf{Poseidon for UML}\footnote{Available at \url{http://www.gentleware.com/products.html}}.
They offer a visual environment to model software, which reduces the complexity of software design.
However, they do not support metrics specification - it is necessary to use other tools, designed for this task.
In the next subsections, we will introduce the leading systems for quantitative analysis of the structural properties of UML models, and put them to test for exploring their features with a real case-study (defined bellow).

One of the tools that we are going to address is SDMetrics\footnote{SDMetrics can be found at \url{http://www.sdmetrics.com/}}, a design measurement tool for UML models.
%SDMetrics is not free so we required an academic license from the staff, which was quickly provided. 
%Its core is open source and is available under the GNU Affero General Public License.
Although its core is open source and available under the GNU Affero General Public License, SDMetrics GUI it is not freely distributed. 
%In order to explore all the system features, SDMetrics staff gently provided us an Academic License which we sincerely would like to thank.
The core functionalities of this system include:
\begin{itemize}
\item the configurable XMI parser for XMI1.0/1.1/1.2/2.0/2.1 input files;
\item the metrics engine to calculate the user-defined design metrics;
\item the rule engine to check the user-defined design rules.
\end{itemize}


The other tool we put to test is the \textsf{Sparx System Enterprise Architect}{\footnote{\url{http://www.sparxsystems.com.au}}}, a team-based modeling environment. 
It embraces the full product development lifecycle, supporting both software design, requirements management, and metrics calculation for Use Case Diagrams.
Thus, it allow to estimate the complexity of the project in a earlier stage, as well as the complexity associated with each actor of the system.
