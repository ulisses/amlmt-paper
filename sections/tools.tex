\section{Tools for UML Metrics Calculation} \label{tools}

Nowadays, it is very common to use tools like \textsf{Visual Paradigm for UML\footnote{Available at \url{http://www.visual-paradigm.com/product/vpuml/}}} or even \textsf{Poseidon for UML}\footnote{Available at \url{http://www.gentleware.com/products.html}} for software application development.
They offer a visual environment to model software, which reduces the complexity of software design.
However, they do not support metrics specification - it is necessary to use other tools, designed for this task.
In the next subsections, we will introduce the leading systems for quantitative analysis of the structural properties of \umlS models, and put them to test for exploring their features with a real case-study.
% (defined bellow).

One of the tools that we are going to address is \sdmetrics\footnote{Available at \url{http://www.sdmetrics.com/}}, a design measurement tool for \umlS models.
Although its core is open source and available under the GNU Affero General Public License, \sdmetricsS GUI it is not freely distributed. 
%The core functionalities of this system include:
%\begin{itemize}
%\item the configurable XMI parser for XMI1.0/1.1/1.2/2.0/2.1 input files;
%\item the metrics engine to calculate the user-defined design metrics;
%\item the rule engine to check the user-defined design rules.
%\end{itemize}
It is a very complete design measurement tool, analyzing a wide range of \umlS diagrams, including Class, Use Case, Activity and Statemachine diagrams, generating several metrics for each type of diagram.

The other tool we put to test is the \textsf{Sparx System Enterprise Architect}{\footnote{Available at \url{http://www.sparxsystems.com.au}}}, a team-based modeling environment. 
It embraces the full product development lifecycle, supporting both software design, requirements management, and metrics calculation for Use Case Diagrams.
Thus, it allow to estimate the complexity of the project in a earlier stage, as well as the complexity associated with each actor of the system.
