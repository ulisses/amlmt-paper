\section{Conclusion} \label{conc}
\entArch\ is a full formal \uml\ specification environment that also supports metrics. 
It is driven to enterprise market, thus oriented to minimize the cost and time spent with the production process, so the metrics evaluation offered is directed to this goal.

As mentioned, its metrics are oriented to available system functionalities and oriented to Use-case diagrams. 
%bad
We consider this a valid approach, but it requires some extra user interaction in the specification of the complexity of a task.
This complexity could be archived by other means if other types of diagrams were also analyzed.
%good
The final results presented are simple and also oriented to the final cost, but if the complexity has been specified with knowledge of the environment, they could estimate very well the final cost of the implemented system.

Summing up, in metrics evaluation, this tool is a good choice to have an estimation of the implementation cost on a formally specified system, but this is its only goal, it cannot provide any kind of analysis about the quality of specification or do any automatic analysis about the system complexity.

%SDMetrics
Adressing now \textit{SDMetrics}, we acknowledge that it is a very versatile tool. It calculates a large set of metrics over a wide range of \uml\ diagrams. Based on this metrics we can try to measure the quality of the software, and also it's complexity.\\
It has an interesting Graphical User Interface, which provides a series of views for the output, from simple \textbf{tables} to \textbf{Histograms}. It finds potencial problems with the modelation and also has the interesting feature of allowing us to define new metrics.

The down side to this tool is the inability of giving the user a plain and simple notion of modelation quality, altough good documentation is provided,  which provide guidelines to the result interpretation.
 

%Comparação entre as diversas ferramentas.