\section{Conclusion} \label{conc}

In general, the software development process follows a systematic approach aiming the achievement of a quality system.
This software quality is not limited to the development of systems which respect several quality standards; it also ensures that the software fulfills all the specified requirements.
Thus, it became essential to include metrics on the software development lifecycle for monitoring its quality.
Most of the existing approaches involve a source code analysis and cannot be applied in earlier stages of the development process.
Applying metrics to \uml\ models enable an early estimate of development efforts, implementation time, complexity and cost of the system under development.
For this paper we focus our research on the most relevant metrics for this models and into testing two tools capable to measure them: \sdmetrics\ and  \entArch\ systems.

%A qualidade na ̃o passa apenas pela obtenc ̧a ̃o de software que respeita as diversas normas e padro ̃es de qualidade, mas tambe ́m pela garantia de que este cumpre todos os requisitos especificados. As medic ̧o ̃es quantitativas associadas a esta problema ́tica revelam-se essenciais em qualquer cieˆncia, o que na ̃o e ́ excepc ̧a ̃o na Engenharia de Software: existe um esforc ̧o cont ́ınuo para encontrar novas propostas, adequadas ao desenvolvimento de software. Com este artigo pretendemos assim mostrar como tirar partido das Linguagens de Modelac ̧a ̃o existentes, e aplicar estas medic ̧o ̃es qualitativas numa fase mais inicial do projecto (aquando a sua modelac ̧a ̃o), de forma a na ̃o centrar a ana ́lise apenas no co ́digo fonte. Apresentamos assim as principais me ́tricas existentes dentro da a ́rea da medic ̧a ̃o de modelos de UML e algumas das principais ferramentas para especificac ̧a ̃o de me ́tricas sobre UML, que combinado com a aplicac ̧a ̃o de me ́tricas tambe ́m ao co ́digo fonte se revela um aliado fundamental para uma melhor gesta ̃o do software produzido.

%SDMetrics
We can conclude that \sdmetrics\ is a versatile tool for calculate a large set of metrics over a wide range of \uml\ diagrams. Based on this metrics we can try to measure the quality and complexity of a software model.
It has an interesting GUI which provides several output views, from simple \emph{tables} to \emph{Histograms}. It finds potential problems with the model and also enables new metrics definition.

The major disadvantage of this tool is the inability of giving the user a plain and simple notion of the model quality, although \sdmetrics\ Manual provides simple tips of how to interpret each kind of metric -- crucial for a correct results reading.
At a glance, \sdmetrics\ results are guidelines for finding the good and the bad points of  \uml\ models, not a full specification of the models quality. 
%Although good documentation is provided,  which provide guidelines to the result interpretation.

On the other hand, \entArch\ is a full formal \uml\ specification environment that also supports metrics calculation oriented to Use Case diagrams. 
It is driven to enterprise market and oriented to minimize the cost and time spent with the production process.

This tool provides simple results and represents a good choice for have an estimation of the implementation cost on a formally specified system. 
With its system wizard, the user can adapt the value of the factors related to the environment and technics complexity for obtain an accurate estimation of the system final cost.
This requires some extra user interaction in the specification of a task complexity and a large knowledge of the system under development.
This complexity could be archived by other means if other types of diagrams were also analyzed.
Summing up, \entArch\  does not provide any kind of analysis about the quality of the model or do any automatic analysis about the system complexity: it estimates the final cost and effort of the project.

\begin{comment}
%bad
We consider this a valid approach, but it requires some extra user interaction in the specification of the complexity of a task.
This complexity could be archived by other means if other types of diagrams were also analyzed.
%good
The final results presented are simple and also oriented to the final cost, but if the complexity has been specified with knowledge of the environment, they could estimate very well the final cost of the implemented system.

Summing up, in metrics evaluation, this tool is a good choice to have an estimation of the implementation cost on a formally specified system, but this is its only goal, it cannot provide any kind of analysis about the quality of specification or do any automatic analysis about the system complexity.
\end{comment}

