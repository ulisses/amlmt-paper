\section{Introduction}

\indent
\par A model is a representation of reality aiming for the simplification of some complex object.

\par Models are built so that we can better understand the system being developed.
They help us to visualize a system as it is or as we need it to be. Models allow us to specify the structure and behavior of a system, they provide the guidance lines/blueprints in constructing a system, and finally, models document the decision taken for a given system;


\par Some models are best described textually, other graphically. All interesting systems exhibit structures that transcend what can be represented in a programming language.


\par A modeling language is a language whose vocabulary and rules focus on the conceptual and physical representation of a system.%%todo

\par Specifying means building models that are precise, unambiguous and complete. The UML addresses the specification of all the important analysis, design and implementation decision. %todo

%\par Unified Modeling Language as a standard language to analyze, design and document software intensive solutions

\par That aside, when assessing a modeling language we might infer on its quality.
\par The software functionality, trustability, usability, efficiency, maintability and portability.

%    * Falar da importância das linguagens de modelação como especificação formal de um projecto e também para ter uma visão global do projecto.

%    *Falar da qualidade do SW e das Linguagens em geral e introduzir o tema de "aferição de qualidade em Linguagens de Modelação"; relacionar com o tema das "métricas".

    * Importância do uso de métricas num projecto: no que consistem, o que medem, o que ajudam a melhorar, etc. - esta parte liga-se com o que sugiro abaixo - sugiro partir de uma definição geral de métricas;\\
    
    
Nowadays, metrics become increasingly essential for Software Engineering: they are crucial for quality assessment and reengineering processes.
In \emph{Forward Engineering} they are used to measure the software quality and estimate cost and effort of software projects\cite{Fenton}.
In the field of \emph{Software Evolution}, they can be both used to identify stable or unstable parts of a system as to determine where refactoring can be or have been applied\cite{Serge}.
They even can be used for assessing the quality and complexity of software systems in \emph{Software Reengineering} or \emph{Reverse Engineering}\cite{43044}.

When focusing on the field of Object-Oriented (OO) systems, many metrics have been proposed for assessing the design of a software system.
However, most of the existing approaches involve the analysis of the source code and cannot be applied in earlier stages of the development process.
In fact, it is not always simple to apply the existing metrics in this earlier stages. 
As the \textsf{Unified Modeling Language}, proposed by Booch, Jacobson and Rumbaugh\cite{USDPuml} has became a standard for expressing OO systems, apply metrics to these models enables an early estimate of development efforts, implementation time, complexity and cost of the system under development. \\

  * Especificar o tipo de métricas sobre o qual nos vamos focar (UML) - o paragrafo acima ja fala 1pc, completar se tiverem+ideias\\
  
  ** estrutura do artigo - proposta a completar\\
  
In this paper, we will introduce and discuss the major existing metrics for UML models, and focus on present a set of tools designed for measure UML projects.
In what follows, Section~\ref{assess} is devoted to the metrics assessment process.
In Section~\ref{metrics} we describe the principal measurements applicable to the most popular UML diagrams.
Then, in Section~\ref{tools} we present some tools designed for apply metrics to UML models and the results of applying them a real case-study.
We conclude in Section~\ref{conc} with a comparsion between the presented tools (.....).
  
    


