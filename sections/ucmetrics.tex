\subsection{Use Case Metrics}

%-- estou a incluir aqui as métricas q temos no nosso artigo, + notas sobre o trabalho d Michele Marchesi


The Use Cases Diagrams are graphical representations of entities which interact with the system under development - the \emph{actors}, and operations that the system must perform for them.
They define a sequence of actions which illustrate a specific way of using the system.

This diagrams are functional specifications, collected at the beginning of a system development process.
They are crucial to an early estimate of the system complexity and its development efforts.\\
  
 
 --- Falta incluir algumas métricas sobre UC e referenciar alguns trabalhos na área;
 
 --- Falta Introduzir textualmente o trabalho d Marchesi para contextualizar a tabela;
 
\begin{center}
\begin{table}[h]
\begin{tabular}{ p{1,5cm} | p{10.5cm}}
\multicolumn{2}{l}{\textbf{Marchesi Metrics}} \\ \hline
\textbf{Metric} & \textbf{Description} \\ \hline
NA & Number of actors of the system. \\ \hline
UC1 & Number of Use Cases in the system. \\ \hline 
UC2 & Number of communications among UC and Actors  \\ \hline 
UC3 & Number of communications among UC and Actores without redundancies \\ \hline 
UC4 & Global complexity of the system \\ \hline 
\end{tabular}
\caption{\small{Use Case Metrics proposed by Michele Marchesi}}
\label{t:ucm}
\end{table}
\end{center}
