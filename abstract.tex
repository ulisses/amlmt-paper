\begin{abstract}
Any traditional engineering field has metrics to rigorously assess the quality of their products.
Engineers know that the output must satisfy the requirements, must comply with the production and market rules, and must be competitive.

Professionals in the new field of software engineering started a few years ago to define metrics to appraise their product: individual programs and software systems.
This concern motivates the need to assess not only the outcome but also the process and tools employed in its development.
In this context, assessing the quality of programming languages is a legitimate objective;
in a similar way, it makes sense to be concerned with models and modeling approaches, as more and more people start the software development process by a modeling phase.

In this paper we introduce and motivate the assessment of models quality in the Software Development cycle.
After the general discussion of this topic, we focus the attention on the most popular modeling language -- the UML -- presenting metrics.
Through a Case-Study, we present and explore two tools.
To conclude we identify what is still lacking in the tools side.

%In the paper we discuss the quality of modeling languages, introducing and motivating the topic, presenting metrics, and comparing tools.
\end{abstract}